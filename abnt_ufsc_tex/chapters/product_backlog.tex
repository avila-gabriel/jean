% ---------------- Product Backlog ----------------
\section{Product Backlog}

\paragraph{Unidade de estimativa:} 1 Story Point (SP) corresponde a 1 hora de trabalho.

\begin{table}[htbp]
  \centering
  \caption{Histórias de usuário}
  \label{tab:backlog}
  \begin{tabular}{cccc}
    \toprule
    Prioridade & ID & Descrição & Estimativa (SP) \\
    \midrule
    Alta  & US-01 & Pesquisar bibliografia inicial & 15 \\
    Alta  & US-02 & Redigir Introdução & 8 \\
    Média & US-03 & Descrever problema e objetivos & 6 \\
    Média & US-04 & Elaborar justificativa e relevância & 5 \\
    Baixa & US-05 & Definir metodologia de pesquisa & 7 \\
    Baixa & US-06 & Planejar cronograma e riscos & 4 \\
    \bottomrule
  \end{tabular}
  \fonte{Elaboração própria.}
\end{table}

\begin{enumerate}[leftmargin=*, label=\arabic*.]
  \item \textbf{Flexibilidade de escopo e adaptação contínua}\\
        Conduzir o desenvolvimento sob abordagem iterativa e adaptativa,
        permitindo ajustes de rota conforme os dados e insumos forem
        disponibilizados por terceiros, respeitando os marcos definidos no
        cronograma.
\end{enumerate}

\subsection{Escopo do Trabalho}\label{escopo-do-trabalho}

\subsubsection{Entradas externas (pré-requisitos)}\label{entradas-externas-prerequisitos}

\begin{table}[htbp]
  \small
  \caption{Entradas externas (pré-requisitos)}
  \label{tab:entradas}
  \centering
  \begin{tabularx}{\linewidth}{@{}>{\RaggedRight\arraybackslash}p{2.5cm}
                                    >{\RaggedRight\arraybackslash}X
                                    >{\centering\arraybackslash}p{2.2cm}
                                    >{\RaggedRight\arraybackslash}p{2.4cm}
                                    >{\RaggedRight\arraybackslash}p{3.3cm}@{}}
    \toprule
    Origem & Entregável & Data prevista & Formato & Observação \\ \midrule
    Eduardo   & Embeddings vetoriais das obras históricas & 31 jul 2025 & Parquet / FAISS / Chroma & Recuperação semântica \\[2pt]
    Franciele & Documento de categorias (3 exemplos cada) & 15 ago 2025 & Markdown & Base p/ regras de categorização \\
    \bottomrule
  \end{tabularx}
  \fonte{Elaboração própria.}
\end{table}